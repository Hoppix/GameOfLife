\documentclass[12pt]{article}


\usepackage{amssymb}
\usepackage{amsmath}
\usepackage[utf8]{inputenc}
\usepackage[ngerman]{babel}
\usepackage{lineno}
\usepackage{listings}
\usepackage[T1]{fontenc}
\usepackage[utf8]{inputenc}
\usepackage{lmodern}
\usepackage{eurosym}
\usepackage{listings}
\usepackage{microtype}
\usepackage{units}
\usepackage{color}
\usepackage{xcolor}
\usepackage{graphicx}
\usepackage{subfigure}
\usepackage{import}
\usepackage{url}
\usepackage{amsthm}
\theoremstyle{plain}

\lstset
{ %
  language=R,                     % the language of the code
  basicstyle=\footnotesize\ttfamily,       % the size of the fonts that are used for the code
  numbers=left,                   % where to put the line-numbers
  numberstyle=\tiny\color{gray},  % the style that is used for the line-numbers
  stepnumber=1,                   % the step between two line-numbers. If it's 1, each line
                                  % will be numbered
  numbersep=5pt,                  % how far the line-numbers are from the code
  backgroundcolor=\color{white},  % choose the background color. You must add \usepackage{color}
  showspaces=false,               % show spaces adding particular underscores
  showstringspaces=false,         % underline spaces within strings
  showtabs=false,                 % show tabs within strings adding particular underscores
  frame=single,                   % adds a frame around the code
  rulecolor=\color{black},        % if not set, the frame-color may be changed on line-breaks within not-black text (e.g. commens (green here))
  tabsize=2,                      % sets default tabsize to 2 spaces
  captionpos=b,                   % sets the caption-position to bottom
  breaklines=true,                % sets automatic line breaking
  breakatwhitespace=false,        % sets if automatic breaks should only happen at whitespace
  title=\lstname,                 % show the filename of files included with \lstinputlisting;
                                  % also try caption instead of title
%  keywordstyle=\color{blue},      % keyword style
%  commentstyle=\color{green},   % comment style
%  stringstyle=\color{blue},       % string literal style
  %escapeinside={\%*}{*)},         % if you want to add a comment within your code
  escapeinside={(*@}{@*)},         
  morekeywords={*,...}            % if you want to add more keywords to the set
} 

\title{default title}
\author{default author \\facculty}
\date{00/00/0000}

\begin{document}
\pagenumbering{arabic}
\maketitle
\centerline{\rule{1.2\linewidth}{.2pt}}
\begin{linenumbers}
\section{TODO}

% PLANUNG
% Projektidee, Anforderungsdefinition, Userstories, ggf. CRC-Cards

% KONZEPT
% Grobbeschreibung für das Gesamtprojekt
% Aufteilung / Modularisierung / Programmstruktur / packages ...
% Prioritätenliste (must items, ..., nice to have)
% Zeitplan (--> Meilensteine)
% Testplan (--> JUnit)
% Darstellung der Beziehungen zwischen den Klassen (--> UML)

% STAND DES PROJEKTS
% was ist fertig?
% was wurde verworfen(warum)?
% wo gibt es macken?
% wie war die workload?
% aufteilung in der gruppe

% FUNKTIONSBESCHREIBUNG
% kurze Bedienungsanleitung (Start, Ende, wichtige Menüs, Parameter, Vorgehensweise, ...)
% ggfs. Screenshots der Programmfenster

% MISC
% Kommentierung der verwendeten Programmierumgebung(en)
% Hinweis auf Probleme, die zu lösen waren
% Kommentierung interessanter Java-Klassen, die benutzt wurden
% Was musste neu erarbeitet werden, wo konnten relevante Komponenten oder Beispiele genutzt werden?

% DOKUMENT MUSS ENTHALTEN
% Überschrift oder Kopfzeile: PTP 2017
% Titel: (z.B. Projekt "Monopoly")
% Autor(en)
% atum der Erstellung / Änderung(en)

\end{linenumbers}
\end{document}
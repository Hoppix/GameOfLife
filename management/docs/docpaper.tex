\documentclass[12pt]{article}


\usepackage{amssymb}
\usepackage{amsmath}
\usepackage[utf8]{inputenc}
\usepackage[ngerman]{babel}
\usepackage{lineno}
\usepackage{listings}
\usepackage[T1]{fontenc}
\usepackage[utf8]{inputenc}
\usepackage{lmodern}
\usepackage{eurosym}
\usepackage{listings}
\usepackage{microtype}
\usepackage{units}
\usepackage{color}
\usepackage{xcolor}
\usepackage{graphicx}
\usepackage{subfigure}
\usepackage{import}
\usepackage{url}
\usepackage{amsthm}
\theoremstyle{plain}

\lstset
{ %
  language=R,                     % the language of the code
  basicstyle=\footnotesize\ttfamily,       % the size of the fonts that are used for the code
  numbers=left,                   % where to put the line-numbers
  numberstyle=\tiny\color{gray},  % the style that is used for the line-numbers
  stepnumber=1,                   % the step between two line-numbers. If it's 1, each line
                                  % will be numbered
  numbersep=5pt,                  % how far the line-numbers are from the code
  backgroundcolor=\color{white},  % choose the background color. You must add \usepackage{color}
  showspaces=false,               % show spaces adding particular underscores
  showstringspaces=false,         % underline spaces within strings
  showtabs=false,                 % show tabs within strings adding particular underscores
  frame=single,                   % adds a frame around the code
  rulecolor=\color{black},        % if not set, the frame-color may be changed on line-breaks within not-black text (e.g. commens (green here))
  tabsize=2,                      % sets default tabsize to 2 spaces
  captionpos=b,                   % sets the caption-position to bottom
  breaklines=true,                % sets automatic line breaking
  breakatwhitespace=false,        % sets if automatic breaks should only happen at whitespace
  title=\lstname,                 % show the filename of files included with \lstinputlisting;
                                  % also try caption instead of title
%  keywordstyle=\color{blue},      % keyword style
%  commentstyle=\color{green},   % comment style
%  stringstyle=\color{blue},       % string literal style
  %escapeinside={\%*}{*)},         % if you want to add a comment within your code
  escapeinside={(*@}{@*)},         
  morekeywords={*,...}            % if you want to add more keywords to the set
} 

\title{default title}
\author{default author \\facculty}
\date{00/00/0000}

\begin{document}
\pagenumbering{arabic}
\maketitle
\centerline{\rule{1.2\linewidth}{.2pt}}
\begin{linenumbers}
\section{Konzept}
-GOGOL (kurz für Game of Game of Life) ist eine Desktop Applikation für das berühmte nullspieler-spiel
Game of Life, welches vom britischen mathematiker John Horton Conway im jahr 1970 entwickelt wurde.
- das grundkonzept des spiels besteht aus einem zweidimensionalen, rechteckigen gitter, wobei jedes feld in diesem gitter einer zelle entspricht
- eine zelle kann pro generationsschritt einen von zwei zuständen haben: lebend, tot
- der zustand einer zelle ist abhängig von den 8 nachbarn die, die zelle umgeben
- bei einem generationsübergang wird nun der zustand einer zelle bestimmt:
- Eine tote Zelle mit genau drei lebenden Nachbarn wird in der Folgegeneration neu geboren.
-Lebende Zellen mit weniger als zwei lebenden Nachbarn sterben in der Folgegeneration an isolation.
- Eine lebende Zelle mit zwei oder drei lebenden Nachbarn bleibt in der Folgegeneration am Leben.
- Lebende Zellen mit mehr als drei lebenden Nachbarn sterben in der Folgegeneration an Überbevölkerung.

Aus diesen einfach regeln enstehen verblüffende strukturen, welche einzigartige verhaltensmuster, bishin zur turing-completeness aufweisen.

Die Grundidee war das entwickeln einer soliden desktop anwendung die weitere nützliche bedienfunktionen hat.
Darüber hinaus soll das programm, neben dem standart spiel noch weitere modi zu implementieren, welche eine abwandlung des "vanilla" GoL bieten.
Zu anfangs geplant waren:
standard conway
colormerge - das verschmelzen von zellen mit farbeigenschaften
colorwar - kampf von zellen mit "teamfarben"
probability of life - random warscheinlichkeit bei der geburt von zellen

zuletzt sollte noch ein lokaler player versus player modus implementiert werden, welcher der anwendung seinen eigentlichen namen verleiht, da so auf dem game of life ein tatsächliches spiel ensteht

Sinn der applikation besteht darin, aus dem game of life einen unterhaltungswert zu gewinnen und somit eine beschäftigung für zwischendurch zu schaffen

\section{Planung}

\subsection{meilensteine}
idee war es die jeweiligen features in meilensteine zu unterteilen sobald eine primitive version des spiel besteht, die meilensteine wurden dann 1:1 auf sprints aufgeteilt die alle 1 oder 2 wochen länge hatten,
je nach vermutung der workload, so enstanden zu nächst folgende meilensteine und sprints:
-grundimplementation: 1. primitive laufende version
-automatisierung: qol- features wie: play/stop, speedregler, load/save, preloaden von strukturen sog. species
-rainbow: custom rules, colormerge, colorwar
-bio: probability of life
-pvp: exterminate, populate
-dokumentation
-website: deployment einer einfachen webseite zum hosten der anwendung als web-applet oder bereitstellen zum download

wir haben gfrundfunktionalitäten priorisiert da höhere funktionalitäten suksessiv auf niedrigere features aufbauen,
höhere funktionen wäre ohne vorherige implementation der vorherigen grundfunktionen nicht lauffähig

\subsection{zeitplan}
die sprints haben zueinander einen critical path, jedoch war theoretisch die abgabe schon nach dem abschließen der automatisierung möglich
für die abgabe war der plan, eine woche vor deadline fertig zu sein um bei auftretenden problemen einen puffer zu haben

\newline
LINK OLD GANT:PNG
der erste entwurf des gant diagrams

wie hat sich die planung im laufe des projekt verändert?
aufgrund von fehlerhaften schätzungen wurden subfeatures und einige meilensteine komplett entfernt oder verschoben

demnach veränderte sich der projektverlauf
siehe auch: was wurde verworfen

\newline
LINK NEW GANTT
letzte aktuelle version der planung

\subsection{testplan}

tests wurden dem jeweiligen sprint zugeteilt und nach fertigstellung der features implementiert
mit ausnahme von rainbow, dort wurde ein test first ansatz verwendet

\subsection{technische beschreibung des systems}

die systemarchitektur liegt dem MVC (Model, View, Controller)-Ansatz zugrunde:
Der View-teil besteht aus einem GUI, sowie einem Gamegrid, welches an das GUI übergeben wird und dieses darstellt.
Das GUI hält das gamegrid lediglich als container und führt keine sondierenden methoden auf dem gamegrid aus.
Als Model-Teil wurden zunächst die Zellen implementiert, diese halten keine logik sondern geben nur ihren status nach außen oder kriegen ihre eigenschaften
von außen gesetzt. Die Zellen werden dann in einem 2 dimensionalen array: der survivalmatrix, gesetzt. Desweiteren gibt es ein Regelwerk hier: Ruler,
dieser hält alle methoden der spielregeln und gibt das cellverhalten vor, welche vom controller dann verwendet werden. Für den PvP modus ist ein Referee zuständig der die 2spielerreglen
verwaltet. Für das preloaden der species wurde ein species model entwickelt, welches die eigenschaften für die SVM hält und in einer specieslibrary verwaltet wird. der Preloader wendet diese auf die SVM an.
Hauptkomponente des Controller-teils stellt der Controller dar, dieser hält alle logikteile, die wiederum doe modelle halten. er hält außerdem als einziger die SVM.
des weiteren kriegt er bei seiner erstelluing das gamegrid und das gui überreicht. daher ist er für die verwaltung zuständig, er stellt verbindung zwischen logik und gui her in dem
er alle listener erstellt und an die buttons bindet.

\newline
LINK CLASS DIAGRAM

als kapselung wurden die klassen in mehrere packages unterteilt und somit ihrer funktionalität unterteile:
frontend:
SLiderUI
EndgameDialog
Gamegrid
LifeGui
PaintImage

backend:
command
controller
player
referee
ruler
saver

cells:
cell
coloredcell
conwaycell
pvpcell

library:
preloader
species
specieslibrary

listener:
buttonlistener
celltogglelistener
gamemodelistener
preloadlistener
speedchangerlistener

\newline
LINK PACKAGE DIAGRAMM

\section{stand des projekts}
% STAND DES PROJEKTS
% was ist fertig?
- Game of Life nach Conway
- Modi: ColorMerge, ColorWar, PvP
- Speichern/Laden des Spiels, laden bestimmter Presets
% was wurde verworfen(warum)?
- Modus: Bio
	- nicht mehr Deterministisch
	- nutzung von Presets, finden von Strukturen nicht möglich
	- kaum Korrellation zu Game of Life
- Modus: PvP - Exterminate
	- gewonnen hätte der Spieler der mehr Zellen des Gegners vernichtet, unabhaengig von seiner eigenen Zellzahl
	- verworfen, da kaum sinnvoll zu bestimmen ob eine Zelle "vernichtet" wurde oder nur abgestorben ist.
- Button changeGridsize
	- Funktionalität ist implementiert
	- Wird in Zukunft vermutlich noch aufgenommen
	- bei Aktuller GUI Struktur kein Platz
- Webscraping
	- geplant war aus http://conwaylife.appspot.com/library die bekannten Presets auszulesen
	- scheitert momentan an der Interpretation von Strukturen, die Zeilen mit ausschließlich toten Zellen enthalten
	- die große Menge an Presets ist für den User schwer übersichtlich darstellbar
	- Wird in Zukunft ggf. noch aufgenommen
- Custom Ruleset
	- geplant war die Regeln für das entstehen/absterben von Zellen anpassbar zu machen
	- wurde verworfen, da Presets nicht mehr sinnvoll verwendbar wären
- Effizientes Berechnen der Zellzustaende
	- wurde aus Zeitmangel und wegen ausreichend guter performance verworfen
	- Ueberlegung war die SVM zeilenweise in threads zu berechnen
	- repaint in stepForward erst am ende des Steps ausfuehren, statt nach jeder berechneten Zelle
% wo gibt es macken?
- SpeedSlider
	- Geschwindigkeit ist logarhitmisch
	- intuitiver wäre eine lineare Funktion
	- aufgrund von geringer Wichtigkeit vorerst nicht angepasst
% wie war die workload?
- Workload von Meilenstein "Rainbow" stark unterschaetzt
	- Unklarheiten in Methodendefinitionen des Backend
	- Methoden des jeweils anderen wurden falsch verstanden und inkorrekt verwendet
- Workload von Meilenstein "PvP" ueberschaetzt
	- grossteil der noetigen Funktionalitaet wurde in "Rainbow" bereits implementiert
% aufteilung in der gruppe
- Kolja Hopfmann
	- Frontend: all
	- Listener: all
	- Backend: Player, Commandhandling
	- Planung: erstellung GANTT diagramm
- Jonas Sander
	- Test: all
	- Backend: Cells, Library, Saver, Ruler, Referee
- Beide
	- Controller, Referee, Planung

% FUNKTIONSBESCHREIBUNG
% kurze Bedienungsanleitung (Start, Ende, wichtige Menüs, Parameter, Vorgehensweise, ...)
- readme
% ggfs. Screenshots der Programmfenster

% MISC
% Kommentierung der verwendeten Programmierumgebung(en)
- verwendete Programme
	- Eclipse Neon
	- IntelliJ IDEA Ultimate 2017.1.2
	- gimp
	- texmaker
	- GanttProject 2.7.1
% Hinweis auf Probleme, die zu lösen waren
- Funktionsweise Java.AWT.Graphics war komisch
% Kommentierung interessanter Java-Klassen, die benutzt wurden
- Es wurde kein Framework verwendet
	- alle relevanten Klassen und Methoden wurden selber entwickelt und implementiert
% Was musste neu erarbeitet werden, wo konnten relevante Komponenten oder Beispiele genutzt werden?
- wir haben uns die Definition von Game of Life angelesen
	- Implementierung, Datentypen und Darstellung wurden selber erarbeitet

% DOKUMENT MUSS ENTHALTEN
% Überschrift oder Kopfzeile: PTP 2017
% Titel: (z.B. Projekt "Monopoly")
% Autor(en)
% atum der Erstellung / Änderung(en)

\end{linenumbers}
\end{document}
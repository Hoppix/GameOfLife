\documentclass[12pt]{article}


\usepackage{amssymb}
\usepackage{amsmath}
\usepackage[utf8]{inputenc}
\usepackage[ngerman]{babel}
\usepackage{lineno}
\usepackage{listings}
\usepackage[T1]{fontenc}
\usepackage[utf8]{inputenc}
\usepackage{lmodern}
\usepackage{eurosym}
\usepackage{listings}
\usepackage{microtype}
\usepackage{units}
\usepackage{color}
\usepackage{xcolor}
\usepackage{graphicx}
\usepackage{subfigure}
\usepackage{import}
\usepackage{url}
\usepackage{amsthm}
\theoremstyle{plain}

\lstset
{ %
  language=R,                     % the language of the code
  basicstyle=\footnotesize\ttfamily,       % the size of the fonts that are used for the code
  numbers=left,                   % where to put the line-numbers
  numberstyle=\tiny\color{gray},  % the style that is used for the line-numbers
  stepnumber=1,                   % the step between two line-numbers. If it's 1, each line
                                  % will be numbered
  numbersep=5pt,                  % how far the line-numbers are from the code
  backgroundcolor=\color{white},  % choose the background color. You must add \usepackage{color}
  showspaces=false,               % show spaces adding particular underscores
  showstringspaces=false,         % underline spaces within strings
  showtabs=false,                 % show tabs within strings adding particular underscores
  frame=single,                   % adds a frame around the code
  rulecolor=\color{black},        % if not set, the frame-color may be changed on line-breaks within not-black text (e.g. commens (green here))
  tabsize=2,                      % sets default tabsize to 2 spaces
  captionpos=b,                   % sets the caption-position to bottom
  breaklines=true,                % sets automatic line breaking
  breakatwhitespace=false,        % sets if automatic breaks should only happen at whitespace
  title=\lstname,                 % show the filename of files included with \lstinputlisting;
                                  % also try caption instead of title
%  keywordstyle=\color{blue},      % keyword style
%  commentstyle=\color{green},   % comment style
%  stringstyle=\color{blue},       % string literal style
  %escapeinside={\%*}{*)},         % if you want to add a comment within your code
  escapeinside={(*@}{@*)},         
  morekeywords={*,...}            % if you want to add more keywords to the set
} 

\title{default title}
\author{default author \\facculty}
\date{00/00/0000}

\begin{document}
\pagenumbering{arabic}
\maketitle
\centerline{\rule{1.2\linewidth}{.2pt}}
\begin{linenumbers}
\section{Konzept}
-GOGOL (kurz für Game of Game of Life) ist eine Desktop Applikation für das berühmte nullspieler-spiel
Game of Life, welches vom britischen mathematiker John Horton Conway im jahr 1970 entwickelt wurde.
- das grundkonzept des spiels besteht aus einem zweidimensionalen, rechteckigen gitter, wobei jedes feld in diesem gitter einer zelle entspricht
- eine zelle kann pro generationsschritt einen von zwei zuständen haben: lebend, tot
- der zustand einer zelle ist abhängig von den 8 nachbarn die, die zelle umgeben
- bei einem generationsübergang wird nun der zustand einer zelle bestimmt:
- Eine tote Zelle mit genau drei lebenden Nachbarn wird in der Folgegeneration neu geboren.
-Lebende Zellen mit weniger als zwei lebenden Nachbarn sterben in der Folgegeneration an Einsamkeit.
- Eine lebende Zelle mit zwei oder drei lebenden Nachbarn bleibt in der Folgegeneration am Leben.
- Lebende Zellen mit mehr als drei lebenden Nachbarn sterben in der Folgegeneration an Überbevölkerung.

Die Grundidee war das entwickeln einer soliden desktop anwendung die weitere nützliche bedienfunktionen hat.
Darüber hinaus soll das programm, neben dem standart spiel noch weitere modi zu implementieren, welche eine abwandlung des "vanilla" GoL bieten.
Zu anfangs geplant waren:
standard conway
colormerge - das verschmelzen von zellen mit farbeigenschaften
colorwar - kampf von zellen mit "teamfarben"
probability of life - random warscheinlichkeit bei der geburt von zellen

zuletzt sollte noch ein lokaler player versus player modus implementiert werden, welcher der anwendung seinen eigentlichen namen verleiht, da so auf dem game of life ein tatsächliches spiel ensteht

\section{Planung}

% PLANUNG
% Projektidee, Anforderungsdefinition, Userstories, ggf. CRC-Cards

% KONZEPT
% Grobbeschreibung für das Gesamtprojekt
% Aufteilung / Modularisierung / Programmstruktur / packages ...
% Prioritätenliste (must items, ..., nice to have)
% Zeitplan (--> Meilensteine)
% Testplan (--> JUnit)
% Darstellung der Beziehungen zwischen den Klassen (--> UML)
\section{stand des projekts}
% STAND DES PROJEKTS
\subsection{Was ist fertig?}
% was ist fertig?
Das klassische Game of Life nach Conway ist vollständig, sowie die alternativen Spielmodi ColorMerge, ColorWar und PvP.
Für jeden dieser Modi sind die Funktionalitäten Speichern/Laden des aktuellen Spielstandes, laden bestimmter Presets, sowie (ausgenommen für den Modus PvP) eine zufällige Initialisierung des Spielfeldes.
\subsection{Verworfen}
% was wurde verworfen(warum)?
Der geplante Modus Propability of Life wurde verworfen. Dieser hätte beinhaltet, dass Zellen abhängig von ihrer Anzahl Nachbarn eine bestimmte Wahrscheinlichkeit haben in der nächsten Generation lebend oder tot zu sein. Das Zellverhalten wäre somit jedoch nicht mehr deterministisch. Eine Nutzung der Presets oder das eigene finden bzw. erstellen solcher wäre in diesem Modus somit unmöglich, womit der entscheidende Aspekt des Game of Life verloren gegangen wäre. Auch wenn die Idee interessant ist, hat sie keinen Bezug zu dem Game of Life und den restlichen Spielmodi.
\newline
Der geplante Modus PvP - Exterminate wurde verworfen. In diesem PvP Modus wäre is das Ziel gewesen möglichst viele Zellen des Gegners zu vernichten. Schon die Definition wann eine Zelle vernichtet oder einfach nur abgestorben ist gestaltet sich als schwierig und wäre für die Spieler schwer verständlich und während des Spiels nicht in realistischer Zeit nachvollziehbar.
\newline
Das geplante Feature Custom Ruleset wurde verworfen. Dieses hätte dem Spieler ermöglicht die Bedingungen, bei wievielen Nachbarn eine Zelle lebend oder tot sein wird anzupassen. Dadurch werden jedoch Presets nicht mehr nutzbar, da diese auf dem Verhalten nach den Conway regeln beruhen, womit ein wichtiges Feature in diesem Modus vom User nicht erwartete Ergebnisse erzeugt hätte.
\newline
Das Feature Change Gridsize ist momentan nicht nutzbar. Die Funktionalität ist bereits implementiert, jedoch ist kein Button auf der Öberfläche implementiert. Grund hierfür ist mangelnder Platz und nicht ausreichend Zeit um die GUI zu refactorn und welchen zu schaffen. In zukünftigen Patches wird dieses Feature vermutlich aufgenommen.
\newline
Das geplante großflächige Bereitstellen von Presets ist momentan noch nicht nutzbar. Geplant war aus http://conwaylife.appspot.com/library die bekannten Presets per Webscraping auszulesen. Dies ist erfolgreich implementiert, redoch wird der code wir einzelne Presets nicht korrekt interpretiert, wodurch diese falsch dargestellt werden.  Des Weiteren gestaltet es sich schwer die große Menge an Presets für den User übersichtlich darzustellen. In zukünftigen Patches wird dieses Feature vermutlich aufgenommen.
\newline
Überlegungen zur Verbesserung der Effizienz bei der Zustandsberechnung der Zellen wurden noch nicht implementiert. Grund hierfür sind mangelnde Zeit und bereits ausreichend gute Performance der Anwendung. In zukünftigen Patches wird die Effizienz vermutlich verbessert.

\subsection{Bekannte Bugs und Probleme}
% wo gibt es macken?
Bisher sind keine Bugs bekannt. 
\newline
Ein bekanntes Problem ist, dass der SpeedSlider eine Exponentielle Funktion für die Geschwindigkeit verwendet. Diese liefert zwar die gewünschte Funktionalität, ist jedoch weniger intuitiv als eine Lineare Funktion. Aufgrund geringer Priorität wird diese Anpassung in zukünftigen Patches durchgeführt.
\subsection{Workload}
% wie war die workload?
Die Workload des Meilensteins "Rainbow" wurde deutlich unterschätzt. Grund hierfür waren unterschiedliche Interpretationen der Methodendefinition. Hier raus resultierten Fehler im Verständnis der vom jeweils anderen geschriebenen Methoden und falsche Anwendung dieser. Anstatt einer Woche waren hier zwei Wochen nötig.
\newline
Die Workload des Meilensteins "PvP" wurde überschätzt. In diesem Meilenstein konnte sein sehr großer Teil der Funktionalität auf dem Modus ColorWar aus dem vorherigem Meilenstein "Rainbow" aufgebaut werden. Anstelle der geplanten zwei Wochen wurde weniger als eine Woche benötigt.
\subsection{Arbeitsaufteilung}
% aufteilung in der gruppe
Die folgenden Aufgaben wurden von Kolja Hopfmann und Jonas Sander gemeinsam durchgeführt:
\newline
Projektplanung, Implementierung der Klassen Controller und Referee
\newline
Die folgenden Aufgaben wurden von Kolja Hopfmann übernommen:
\newline
Projektmanagement, Implementierung der Packages Frontend und Listener, Implementierung der Klasse Player und des Commandhandling im Backend
\newline
Die folgenden Aufgaben wurden von Jonas Sander übernommen:
\newline
Implementierung der Packages Testing, Cells und Library, sowie die Implementierung der Klassen Saver und Ruler im Backend.
\section{Funktionsbeschreibung}
% FUNKTIONSBESCHREIBUNG
% kurze Bedienungsanleitung (Start, Ende, wichtige Menüs, Parameter, Vorgehensweise, ...)
- readme
% ggfs. Screenshots der Programmfenster

% MISC
% Kommentierung der verwendeten Programmierumgebung(en)
- verwendete Programme
	- Eclipse Neon
	- IntelliJ IDEA Ultimate 2017.1.2
	- gimp
	- texmaker
	- GanttProject 2.7.1
% Hinweis auf Probleme, die zu lösen waren
- Funktionsweise Java.AWT.Graphics war komisch
% Kommentierung interessanter Java-Klassen, die benutzt wurden
- Es wurde kein Framework verwendet
	- alle relevanten Klassen und Methoden wurden selber entwickelt und implementiert
% Was musste neu erarbeitet werden, wo konnten relevante Komponenten oder Beispiele genutzt werden?
- wir haben uns die Definition von Game of Life angelesen
	- Implementierung, Datentypen und Darstellung wurden selber erarbeitet

% DOKUMENT MUSS ENTHALTEN
% Überschrift oder Kopfzeile: PTP 2017
% Titel: (z.B. Projekt "Monopoly")
% Autor(en)
% atum der Erstellung / Änderung(en)

\end{linenumbers}
\end{document}
\documentclass[12pt]{article}


\usepackage{amssymb}
\usepackage{amsmath}
\usepackage[utf8]{inputenc}
\usepackage[ngerman]{babel}
\usepackage{lineno}
\usepackage{listings}
\usepackage[T1]{fontenc}
\usepackage[utf8]{inputenc}
\usepackage{lmodern}
\usepackage{eurosym}
\usepackage{listings}
\usepackage{microtype}
\usepackage{units}
\usepackage{color}
\usepackage{xcolor}
\usepackage{graphicx}
\usepackage{subfigure}
\usepackage{import}
\usepackage{url}
\usepackage{amsthm}
\theoremstyle{plain}

\lstset
{ %
  language=R,                     % the language of the code
  basicstyle=\footnotesize\ttfamily,       % the size of the fonts that are used for the code
  numbers=left,                   % where to put the line-numbers
  numberstyle=\tiny\color{gray},  % the style that is used for the line-numbers
  stepnumber=1,                   % the step between two line-numbers. If it's 1, each line
                                  % will be numbered
  numbersep=5pt,                  % how far the line-numbers are from the code
  backgroundcolor=\color{white},  % choose the background color. You must add \usepackage{color}
  showspaces=false,               % show spaces adding particular underscores
  showstringspaces=false,         % underline spaces within strings
  showtabs=false,                 % show tabs within strings adding particular underscores
  frame=single,                   % adds a frame around the code
  rulecolor=\color{black},        % if not set, the frame-color may be changed on line-breaks within not-black text (e.g. commens (green here))
  tabsize=2,                      % sets default tabsize to 2 spaces
  captionpos=b,                   % sets the caption-position to bottom
  breaklines=true,                % sets automatic line breaking
  breakatwhitespace=false,        % sets if automatic breaks should only happen at whitespace
  title=\lstname,                 % show the filename of files included with \lstinputlisting;
                                  % also try caption instead of title
%  keywordstyle=\color{blue},      % keyword style
%  commentstyle=\color{green},   % comment style
%  stringstyle=\color{blue},       % string literal style
  %escapeinside={\%*}{*)},         % if you want to add a comment within your code
  escapeinside={(*@}{@*)},         
  morekeywords={*,...}            % if you want to add more keywords to the set
} 

\title{default title}
\author{default author \\facculty}
\date{00/00/0000}

\begin{document}
\pagenumbering{arabic}
\maketitle
\centerline{\rule{1.2\linewidth}{.2pt}}
\begin{linenumbers}
\section{TODO}

% PLANUNG
% Projektidee, Anforderungsdefinition, Userstories, ggf. CRC-Cards

% KONZEPT
% Grobbeschreibung für das Gesamtprojekt
% Aufteilung / Modularisierung / Programmstruktur / packages ...
% Prioritätenliste (must items, ..., nice to have)
% Zeitplan (--> Meilensteine)
% Testplan (--> JUnit)
% Darstellung der Beziehungen zwischen den Klassen (--> UML)

% STAND DES PROJEKTS
% was ist fertig?
- Game of Life nach Conway
- Modi: ColorMerge, ColorWar, PvP
- Speichern/Laden des Spiels, laden bestimmter Presets
% was wurde verworfen(warum)?
- Modus: Bio
	- nicht mehr Deterministisch
	- nutzung von Presets, finden von Strukturen nicht möglich
	- kaum Korrellation zu Game of Life
- Modus: PvP - Exterminate
	- gewonnen hätte der Spieler der mehr Zellen des Gegners vernichtet, unabhaengig von seiner eigenen Zellzahl
	- verworfen, da kaum sinnvoll zu bestimmen ob eine Zelle "vernichtet" wurde oder nur abgestorben ist.
- Button changeGridsize
	- Funktionalität ist implementiert
	- Wird in Zukunft vermutlich noch aufgenommen
	- bei Aktuller GUI Struktur kein Platz
- Webscraping
	- geplant war aus http://conwaylife.appspot.com/library die bekannten Presets auszulesen
	- scheitert momentan an der Interpretation von Strukturen, die Zeilen mit ausschließlich toten Zellen enthalten
	- die große Menge an Presets ist für den User schwer übersichtlich darstellbar
	- Wird in Zukunft ggf. noch aufgenommen
- Custom Ruleset
	- geplant war die Regeln für das entstehen/absterben von Zellen anpassbar zu machen
	- wurde verworfen, da Presets nicht mehr sinnvoll verwendbar wären
- Effizientes Berechnen der Zellzustaende
	- wurde aus Zeitmangel und wegen ausreichend guter performance verworfen
	- Ueberlegung war die SVM zeilenweise in threads zu berechnen
	- repaint in stepForward erst am ende des Steps ausfuehren, statt nach jeder berechneten Zelle
% wo gibt es macken?
- SpeedSlider
	- Geschwindigkeit ist logarhitmisch
	- intuitiver wäre eine lineare Funktion
	- aufgrund von geringer Wichtigkeit vorerst nicht angepasst
% wie war die workload?
- Workload von Meilenstein "Rainbow" stark unterschaetzt
	- Unklarheiten in Methodendefinitionen des Backend
	- Methoden des jeweils anderen wurden falsch verstanden und inkorrekt verwendet
- Workload von Meilenstein "PvP" ueberschaetzt
	- grossteil der noetigen Funktionalitaet wurde in "Rainbow" bereits implementiert
% aufteilung in der gruppe
- Kolja Hopfmann
	- Frontend: all
	- Listener: all
	- Backend: Player, Commandhandling
	- Planung: erstellung GANTT diagramm
- Jonas Sander
	- Test: all
	- Backend: Cells, Library, Saver, Ruler, Referee
- Beide
	- Controller, Referee, Planung

% FUNKTIONSBESCHREIBUNG
% kurze Bedienungsanleitung (Start, Ende, wichtige Menüs, Parameter, Vorgehensweise, ...)
- readme
% ggfs. Screenshots der Programmfenster

% MISC
% Kommentierung der verwendeten Programmierumgebung(en)
- verwendete Programme
	- Eclipse Neon
	- IntelliJ IDEA Ultimate 2017.1.2
	- gimp
	- texmaker
	- GanttProject 2.7.1
% Hinweis auf Probleme, die zu lösen waren
- Funktionsweise Java.AWT.Graphics war komisch
% Kommentierung interessanter Java-Klassen, die benutzt wurden
- Es wurde kein Framework verwendet
	- alle relevanten Klassen und Methoden wurden selber entwickelt und implementiert
% Was musste neu erarbeitet werden, wo konnten relevante Komponenten oder Beispiele genutzt werden?
- wir haben uns die Definition von Game of Life angelesen
	- Implementierung, Datentypen und Darstellung wurden selber erarbeitet

% DOKUMENT MUSS ENTHALTEN
% Überschrift oder Kopfzeile: PTP 2017
% Titel: (z.B. Projekt "Monopoly")
% Autor(en)
% atum der Erstellung / Änderung(en)

\end{linenumbers}
\end{document}